\documentclass{article}
\usepackage[utf8]{inputenc}
\usepackage[letterpaper, portrait, margin=1in]{geometry}

\title{BTM710 Assignment 1}
\author{Md Iztiba Hasan, Phu Anh Pham, TJ LeBlanc, Andy Mai, Bo Wei Yao}
\date{September 16 2022}

\begin{document}

\maketitle

\section*{Problems addressed and motivation of the paper - Jay and Bo}
The study cites multiple instances where children were exposed to potentially fatal conditions, including seizures, hypotonia, or acidosis because their parents failed to monitor them adequately.
As a result, the study suggests a wearable sensor that can detect the infant's health status and notify the parents.The initiative has received widespread support from hospitals and parents since it offers hope for solving the issue. The purpose of the publication is to make the sensor device more widely known. In addition to the sensor's essential features, the paper assesses the possibilities and advancements in baby health monitoring technologies. \cite[p. 3723, p.3724]{zhu2015wearable}

\section*{Similar problems in previous work - Andy}
Recently, parents have been turning to new technologies for the monitoring of their children��s health. Children with heart issues will use these devices by attaching gel pads to the infant��s chest to monitor the heart (Zhihua Zhu, 2014, pp. 2-3). These results can show on an electrocardiogram or EGC, if the baby is having any irregular heart activity. In previous works, this method has shown a great way to monitor an infant��s heart activity for any irregularities and allows doctors to get an early read on any issues that may arise (Zhihua Zhu, 2014, p. 5). However previous research has shown issues may arise such as the baby not liking the feel of metal wires or an allergic reaction to the gel contact points (Zhihua Zhu, 2014, p. 9).
The second issue that arose in previous research has been infant respiration monitoring. Generally, there are 2 types of sensors, contact and non-contact (Zhihua Zhu, 2014, p. 11). Contact is attached to the patient��s body to monitor airflow and respiratory sounds (Zhihua Zhu, 2014, pp. 10-12). Contactless use CO2 sensors, which have shown to be effective at giving early warning signs for further diagnostics. Previous work has also shown they are subject to interference from the environment (Zhihua Zhu, 2014, p. 11). The sensor is non-contact and such researchers had to manually use contact methods to get a better read on infant respiration rhythms (Zhihua Zhu, 2014, p. 11). 


\section*{New ideas explored in the research - Md and TJ}
The study explores innovative ways that sensor technology can be used to help aid in the development of infants. For instance, clinicians could use wearable sensors to monitor their internal systems and diagnose disorders such as cystic fibrosis with relative ease (page 3743), and parents could use them to monitor their physiological parameters to keep track of their nutritional needs (3744) along with being able to detect and prevent life threatening events such as falls, drowning and even child abuse (page 3743). Additionally, technological limitations are discussed, with the primary issues being the usage of poor power supplies such as batteries that only last up to 2 hours (page 3741), and the mediocre transmission range and network reliability of current Wireless Sensing Technologies due to sensor node limitations, high power consumption and the unpredictable nature of infrastructure oriented wireless networks in terms of coverage (page 3741). 



\bibliographystyle{apalike}
\bibliography{ref}

\end{document}
